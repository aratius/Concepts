\documentclass[a4paper]{article}
\setlength{\parindent}{1em}  % 一文字分インデント
\usepackage[utf8]{inputenc}
\usepackage{indentfirst}
\title{Worship}
\author{Arata Matsumoto}
\date{April 2023}
\begin{document}
\maketitle
\section{概要}
集団行動パフォーマンスにおいて、その動きを高い精度で統率することはパフォーマンスのクオリティに直結する。通常それは個々のポテンシャルや、全体練習の積み重ねによって向上する。そのような手順が煩雑に感じた私は、エンジニアリングによる解決方法を模索すべく、プロジェクトを開始した。

当パフォーマンスにおいてパフォーマーは演技直前に観覧客の中から募集する。選ばれた観覧客はその場でパフォーマーへと立場が逆転する。パフォーマーはアイマスクとヘッドホンを装着し、外界の情報を遮断された状態でパフォーマンスする。パフォーマーにリアルタイムに振り付け指示を出すための専用ソフトウェアを用い、ソフトウェアの操作者(Controller = Human Jockey)はアプリケーションを即興操作し、パフォーマンスを構築していく。

パフォーマーの周囲にはLEDバーが建てられており、振り付けにリアクティブなエフェクトを生成する。また、Max/MSPによる音楽生成アルゴリズムにより、パフォーマンスにリアクティブな音楽が自動生成される。一般的なリアルタイムレンダリングを用いたVJが音楽に呼応する映像を求められるのに対し(オーディオリアクティブな映像)、当パフォーマンスにおいてはパフォーマンスに呼応する音楽を生成している(パフォーマンスリアクティブな音楽)。さらにその生成された音楽に呼応する映像を生成するという手法を取れば、音・映像・光・振り付けが複雑に絡み合った即興パフォーマンスを構築することも可能である。

\end{document}