\documentclass[a4paper]{article}
\setlength{\parindent}{1em}  % 一文字分インデント
\usepackage[utf8]{inputenc}
\usepackage{indentfirst}
\title{Counterintuitive}
\author{Arata Matsumoto}
\date{April 2023}
\begin{document}
\maketitle
\section{概要}
Counterintuitiveは、現代物理学によって覆された時間概念の視覚化を目的としたインスタレーション作品である。

五つのモニタが空間的に配置され、それぞれがアームによって移動、回転する。鑑賞者からモニタの各ピクセルまでの物理的距離が離れるほど光の到達時間は長くなる。当作品においては光速度を1m/sとおくことによってこれを誇張し、理解を促した。

これにより、鑑賞者は他者と共有している現在という体験が、光速度に対して十分に狭い地球という枠組みにおいて近似された幻に過ぎないことに気づく。固定化された時間の絶対的性質が覆された時、我々は真に時間と向き合うことを求められる。

時間の本質的な姿というのは、我々の生活圏からは認識することが難しい。人生とは現在の積み重ねからなる体験の連続であり、それらが堆積することで形而上的時間を形成する。これこそが我々の知覚する時間の姿であり、我々にとってはむしろこちらの方が大切なのではないか。

科学は日々進歩している。科学により証明された事実に翻弄されず最良の選択をするため、常に方法的懐疑を以て生きる姿勢こそが必要なのではないかと考える。

\end{document}