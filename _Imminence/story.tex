\documentclass[a4paper]{article}
\setlength{\parindent}{1em}  % 一文字分インデント
\usepackage[utf8]{inputenc}
\usepackage{indentfirst}
\usepackage{listings}
\title{Immience -story}
\author{Arata Matsumoto}
\date{April 2023}
\begin{document}
\maketitle
\section{概要}
展望台、夜の海の堤防沿い、崖、鰐の水槽――そういった場所の近くに身を置くと、時折私は不思議な感覚に襲われる。

「自分が望めば、今すぐに死ぬことができる。」

そうした状況に陥ると、死は妙なリアリティを帯び、私の体を包み込む。これは一種の臨死体験と呼べるのではないか。安全な臨死体験。

なぜこんな感情に襲われるのか。私は現在にしか存在しないのであって、未来のことはわからない。突然気が狂って自ら死を選ぶようなことはしない、そう断言する自信はない。たとえそれが5秒後の未来であってもだ。断言できないのは、潜在的に死に惹かれているのではなかろうか。死が持つ不思議な引力に引き寄せられている。実際、進化論において死は種の遺伝子を汚染しないための合理的選択として位置付けられている。我々の遺伝子は、死に引き寄せられるようにプログラムされているのだ。

私は昨年まで電車通学をしていた。駅のホームに立ち、電車を待つ。電車が目の前を通り過ぎるその直前にまたあの感覚に襲われる。それは毎日のルーティンと化していた。繰り返される「安全な臨死体験」を通じて、私は常に死を意識しながら生活した。

昨年末からは大学付近に引っ越したので、それからは電車に乗る機会も少なくなった。私と死の距離はだんだんと開いていき、あの感覚に襲われる機会は自ずと減っていった。

話は変わるが、私は普段プログラマとして在宅勤務をしている。毎日ラップトップを開き、淡々と与えられたタスクをこなす。そんな日々が続いているうちに、ひとつ気がついたことがある。

職業柄日常的に使うツールに、CLI(Command Line Interface)がある。これはコンピュータをテキストベースのコマンドで操作するためのツールであり、ファイル管理やプログラムの実行などのタスクを行うのに便利なものである。ここで実行できるコマンドのひとつに、

\begin{lstlisting}{language=bash}
  sudo rm -rf /
\end{lstlisting}

というものがある。これはファイルやフォルダを管理者権限で削除するためのコマンドであり、その対象としてルート(コンピュータの最上位ディレクトリ)を指定している。つまりこのコマンドを実行すると、コンピュータ内のすべてのファイル(システムファイルを含む)が削除される※1。

これはコンピュータの死と言えるのではないか。さらにプログラマの業務にコンピュータは必須であり、それが動かないとなるとプログラマも死んだことになる。

私はこのコマンドを恐る恐るCLIに入力してみた。すると途端にあの感情が湧き上がってきた。エンターキーを押すとこのコマンドは即座に実行される。コンピュータをいとも簡単に死に至らしめることができる。数秒のちに、私は無事に実行前のコマンドをデリートキーで削除した。なお、エンターキーとデリートキーはキーボード配列上隣り同士に配置されている。

それ以来、私は毎日のルーティンとして例のコマンドを入力しては消すという行為を行っている。繰り返される「安全な臨死体験」を通じて、私は再び死を意識しながら生活するようになった。

\end{document}