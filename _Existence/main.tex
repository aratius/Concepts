\documentclass[a4paper]{article}
\setlength{\parindent}{1em}  % 一文字分インデント
\usepackage[utf8]{inputenc}
\usepackage{indentfirst}
\title{Existence}
\author{Arata Matsumoto}
\date{April 2023}
\begin{document}
\maketitle
\section{概要}
流れ行くレシートとそれに印刷されるバーコードは、過去を生きた名もない人生のアーカイブとしての役割を持つ。彼らは皆、至って普通の人生を全うした。しかし今となっては、誰も彼らの名前を知らない。彼らがかつて、この地球上のどこかに存在したことにすらも気づかない。

打って変わって、時には歴史に名を残すような人生もある。彼らの存在は、あらゆる文献によって後世まで広く認知されることとなる。しかしそれも、それを認知する人間という種族が存続する限りである。

つまり本質的には、個人の存在はその功績には依存しない。全てはいずれ忘れ去られるからである。つまり存在とは、他からの認知・承認を必要としないものであると考える。

当作品は、インタラクティブ要素を含む空間インスタレーション作品である。

厚生労働省が発表している日本人の平均寿命データを実装したモデルが、ランダムな寿命データをリアルタイムに出力する。寿命の年数分のライフイベントを単体の英語動詞として割り当て、人生の経過割合に応じて動詞、性別、年齢をバーコードデータにエンコードする。

それらのバーコードをレシートプリンタでプリントする。スキャナでバーコードを読み取ることで、動詞、性別、年齢がデコードされ、それに対応した自動音声によって読み上げられる。

\end{document}